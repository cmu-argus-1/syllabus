\documentclass[11pt,letterpaper]{article}

\usepackage[margin=1in]{geometry}
\usepackage{termcal}
\usepackage{enumitem}
\usepackage[colorlinks=true, allcolors=blue]{hyperref}
\usepackage{color}
\usepackage{multirow}
\usepackage{multicol}

\newcommand{\todo}[1]{\textcolor{red}{TODO: #1}}

\title{16/18-873: Spacecraft Design-Build-Fly Lab}
\author{Fall 2023 -- Spring 2024}
\date{}

\begin{document}

\maketitle

\section*{Course Description}

Spacecraft design is a truly interdisciplinary subject that draws from every branch of engineering. This capstone design class brings together the material from prior classes in a way that emphasizes the interactions between disciplines and demonstrates how some of the more theoretical topics are synthesized in the practical design of a spacecraft. The class will design, build, and test a small satellite that addresses objectives and requirements posed at the beginning of the course sequence. Students will work in subsystem teams, each focusing on some aspect of the spacecraft, but exposed to many different disciplines and challenges. Practical, hands-on engineering skills will be emphasized, along with fabrication and testing of physical hardware and the creation of thorough documentation.

\section*{Instructors}

\begin{center}
\begin{tabular}{l l}
	Prof. Zac Manchester & \textbf{Email:} \href{mailto:zacm@cmu.edu}{zacm@cmu.edu} \\
	Prof. Brandon Lucia & \textbf{Email:} \href{mailto:blucia@cmu.edu}{blucia@cmu.edu}
	\\
	TA: Brad Denby & \textbf{Email:} \href{mailto:bdenby@cmu.edu}{bdenby@cmu.edu}
	\\
	TA: Neil Khera & \textbf{Email:} \href{nkhera@andrew.cmu.edu}{nkhera@andrew.cmu.edu}
\end{tabular}
\end{center}

\section*{Learning Objectives}

The goal of this course is to give students hands-on experience designing and building small spacecraft subsystems and integrating them into a CubeSat. Throughout this course, students will:
\begin{enumerate}
	\item Understand how the design and integration of a system whose performance depends on the success of many interacting subsystems.
	\item Work within a small team to fabricate, and test hardware and software through rapid design iteration.
	\item Coordinate with other teams to integrate subsystems into a complete spacecraft.
	\item Gain exposure to the complete life cycle of a small satellite mission.
\end{enumerate}


\section*{Logistics}

The course will involve designing and building hardware in small teams. Class time will be used primarily for weekly team meetings and consulting time to meet with the instructors.

\begin{itemize}
	\item Lectures on selected topics will be held at 3:40 on Mondays, followed by consulting hours.
	\item All-hands meetings will be held at 3:40 on Wednesdays, followed by consulting hours.
	\item Sub team meetings will be held once per week at times coordinated with the instructor.
	\item Attendance of weekly team meetings is mandatory.
	\item Slack will be used for coordination between teams and instructors. All students will be added to the ``SpacecraftDesignBuildFlyLab'' slack channel.
	\item GitHub will be used to manage project files for all teams.
\end{itemize}

\section*{Assignments and Exams}

There will be no exams in this course. Evaluation will be based on participation, contribution to design and fabrication work, and final documentation from each team.

\section*{Grading}

Grading will be based on:
\begin{itemize}
	\item 25\% Participation and attendance of team meetings
	\item 25\% Individual technical contributions quantified by git commit history and peer surveys
	\item 15\% Completeness and quality of documentation
	\item 10\% Outcome of design review
\end{itemize}


\section*{Learning Resources}

There is no textbook required for this course. Video recordings of lectures and lecture notes will be posted online. Additional references for further reading will be provided with each lecture.

\section*{Course Policies}

\textbf{Attendance:} This is a team-based course. In order to coordinate work among teams, participation in weekly meetings is required. If you are unable to be present at a meeting, you must notify the instructors and ensure that your teammates are prepared to present your work.

\medskip
\noindent
\textbf{Accommodations for Students with Disabilities:} If you have a disability and are registered with the Office of Disability Resources, I encourage you to use their online system to notify me of your accommodations and discuss your needs with me as early in the semester as possible. I will work with you to ensure that accommodations are provided as appropriate. If you suspect that you may have a disability and would benefit from accommodations but are not yet registered with the Office of Disability Resources, I encourage you to contact them at \href{mailto:access@andrew.cmu.edu}{access@andrew.cmu.edu}.

\medskip
\noindent
\textbf{Statement of Support for Students' Health \& Well-Being:} Take care of yourself. Do your best to maintain a healthy lifestyle this semester by eating well, exercising, avoiding drugs and alcohol, getting enough sleep, and taking some time to relax. This will help you achieve your goals and cope with stress.

\medskip
\noindent
If you or anyone you know experiences any academic stress, difficult life events, or feelings like anxiety or depression, we strongly encourage you to seek support. Counseling and Psychological Services (CaPS) is here to help: call 412-268-2922 and visit \href{http://www.cmu.edu/counseling}{http://www.cmu.edu/counseling}. Consider reaching out to a friend, faculty, or family member you trust for help getting connected to the support that can help.

\medskip
\noindent
\textit{If you or someone you know is feeling suicidal or in danger of self-harm, call someone immediately, day or night:}

\textit{CaPS: 412-268-2922}

\textit{Re:solve Crisis Network: 888-796-8226}

\medskip
\noindent
\textit{If the situation is life threatening, call the police:}

\textit{On campus: CMU Police: 412-268-2323}

\textit{Off campus: 911}


\section*{Tentative Fall 2023 Schedule}

\begin{tabular}{c|c|c|c}
	Week & Dates & Topics & Assignments \\
	\hline
	\multirow{2}{*}{1} & Jan 17 & Course Overview, \& Dynamics Intro & Survey \\
	 & Jan 19 & Stability, Discrete-Time Dynamics &  HW0 Out\\
	\hline
	\multirow{2}{*}{2} & Jan 24 & Optimization Intro & HW0 Due \\
	 & Jan 26 & Numerical Optimization Pt. 1 & HW1 Out \\
	\hline
	\multirow{2}{*}{3}  & Jan 31 & Numerical Optimization Pt. 2 \& Optimal Control Intro &  \\
	 & Feb 2 & Pontryagin, Shooting Methods, \& LQR Intro &  \\
	\hline
	\multirow{2}{*}{4}  & Feb 7 & LQR as a QP \& Riccati Equation & HW 1 Due \\
	 & Feb 9 & \textcolor{red}{No Class} & HW 2 Out \\
	\hline
	\multirow{2}{*}{5}  & Feb 14 & Dynamic Programming \& Intro to Convexity & \\
	 & Feb 16 & Convex Model-Predictive Control &  \\
	\hline
	\multirow{2}{*}{6}  & Feb 21 & Intro to Trajectory Optimization, Iterative LQR, \& DDP & HW2 Due \\
	 & Feb 23 & DDP with Constraints and Free Final Time & HW3 Out \\
	\hline
	\multirow{2}{*}{7}  & Feb 28 & Direct Trajectory Optimization, Collocation, \& SQP & \\
	 & Mar 2 & Attitude Intro: SO(3) \& Quaternions & \\
	\hline
	\multirow{2}{*}{8}  & Mar 7 & \textcolor{red}{No Class} & \\
	 & Mar 9 & \textcolor{red}{No Class} &   \\
	\hline
	\multirow{2}{*}{9}  & Mar 14 & Optimizing with Attitude & HW3 Due \\
	 & Mar 16 & LQR with Attitude, Quadrotors, \& Contact Intro & HW4 Out \\
	\hline
	\multirow{2}{*}{10}  & Mar 21 & Trajectory Optimization for Hybrid Systems &  \\
	 & Mar 23 & Data-Driven Methods \& Iterative Learning Control &   \\
	 \hline
	\multirow{2}{*}{11}  & Mar 28 & Stochastic Optimal Control \& LQG & HW4 Due \\
	 & Mar 30 & Robust Control \& Minimax DDP &   \\
	 \hline
	\multirow{2}{*}{12}  & Apr 4 & RL from an Optimal Control Perspective &   \\
	 & Apr 6 & Practical Tips \& Tricks, Control History &   \\
	 \hline
	\multirow{2}{*}{13}  & Apr 11 & Case Study: How to Land a Rocket &  \\
	 & Apr 13 & \textcolor{red}{No Class} &   \\
	 \hline
	\multirow{2}{*}{14}  & Apr 18 & Case Study: How to Drive a Car &  \\
	 & Apr 20 & Case Study: How to Walk &   \\
	 \hline
	\multirow{2}{*}{14}  & Apr 25 & Project Presentations &  \\
	 & Apr 27 & Project Presentations &   \\
\end{tabular}


\end{document}
